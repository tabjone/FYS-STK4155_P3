\section{Conclusion}
\begin{comment}
State your main findings and interpretations. 
Try as far as possible to present perspectives for future work. 
Try to discuss the pros and cons of the methods and possible improvements.
\end{comment}

For the Neural Network we found that the simpler networks performed better on
this problem. Specifically a 3 layer network trained using RMSprop, with ReLu
activation functions performed best in terms om the MSE against the analytical
solution. But we found that the explicit solver performed better, with about
two-three orders of magnitude lower MSE. This is mostly because of the boundary
conditions, as there is no implementation in the networks to penalize high MSE
on the boundaries. For future work we could create a cost function for the
network as described in the previous section, by looking at the minimization
problem of the heat equation and the boundary conditions.

We also found that the time the already trained network used to predict a
solution was about two orders of magnitude slower than the time used for the
explicit solver. This indicates that for such a simple problem there is no
benefits to using a Neural Network. For future work one could also look at
implicit schemes such as Crank-Nicolson, which is stable for all $\Delta x$ and
$\Delta t$ and could possibly solve even faster. 

One could imagine that as the complexity of the problem increased the Neural
Network would perform better compared to the explicit scheme. Therefore it
would be beneficial to look at the heat equation in 2D and 3D. And if we have a
cost function that is not based on an analytical solution, we could train the
network with ICs as parameters and see if the network can predict for ICs that
it has not been trained on. In that case one could solve problems without
closed-form solutions.


