\section{Theory and Methods}

\subsection{Heat equation}
We will look at the heat equation in one dimension.
\begin{equation*}
    \frac{\partial^2 u(x,t)}{\partial x^2} =\frac{\partial u(x,t)}{\partial t},
    t\geq 0, x\in [0,L]
\end{equation*}
or
\begin{equation*}
    u_{xx} = u_t,
\end{equation*}
with initial conditions
\begin{equation*}
    u(x,0)= \sin{(\pi x)} \hspace{0.5cm} 0 \leq x \leq L,
\end{equation*}
and with $L=1$ being the length of the $x$-region of interest. The 
boundary conditions are
\begin{equation*}
    u(0,t)= 0 \hspace{0.5cm} t \ge 0,
\end{equation*}
and
\begin{equation*}
    u(L,t)= 0 \hspace{0.5cm} t \ge 0.
\end{equation*}

\subsubsection{Closed-form solution}
The analytical solution to this problem is derived in (Tveito and
Winther, 2005, pp. 90-92)\cite{tveito}, so we will just present a small
summary of the method here. First we assume that $u(x,t)$ is linear and homogeneous and 
that the equation is separable in the form of $u_k(x,t)=X_k(x)T_k(t)$, where
$k$ refers to it being a particular solution. We can then solve the equation 
by separation of variables to find
\begin{equation*}
    u(x,t)_k=e^{-(k\pi/L)^2t}\sin{\left(\frac{k\pi}{L}x\right)},\ k=1,2,3,..
\end{equation*}
This will then give the family ${u_k}$ of particular solutions. We assume then
that $f(x)$ can be written as a linear combination of the eigenfunctions of
$X(x)$, so that
\begin{equation*}
    f(x)=\sum_{k=1}^{N} c_k \sin{\left(\frac{k\pi}{L}x\right)},
\end{equation*}
where $c_k$ is some constant. Then it follows by linearity that
\begin{equation*}
u(x,t)=\sum_{k=1}^N c_ke^{-(k\pi/L)^2t}\sin{\left(\frac{k\pi}{L}x\right)}.
\end{equation*}

Now inserting that $f(x)=\sin(\pi x)$ and $L=1$ it is easy to see that this
gives $c_1=1$ and all other constants $c_k=0$. This gives us the \textbf{closed-form
solution} of
\begin{equation*}
    u(x,t)=e^{-\pi^2 t}\sin{\left(\pi x\right)}.
\end{equation*}

\subsubsection{Numerical approximation}
We will solve this by so called FTCS-scheme (Forward Time Central Space) and
use the Forward-Euler method for moving in time. This results in

\begin{equation*}
u_t\approx \frac{u(x,t+\Delta t)-u(x,t)}{\Delta t}=\frac{u(x_i,t_j+\Delta t)-u(x_i,t_j)}{\Delta t}
\end{equation*}
and
\begin{equation*}
u_{xx}\approx \frac{u(x+\Delta x,t)-2u(x,t)+u(x-\Delta x,t)}{\Delta x^2},
\end{equation*}
or
\begin{equation*}
    u_{xx}\approx \frac{u(x_i+\Delta x,t_j)-2u(x_i,t_j)+u(x_i-\Delta x,t_j)}{\Delta x^2}.
\end{equation*}
And to simplify notation, let
\begin{equation*}
    u(x_i, t_j + \Delta t)\rightarrow u_{i,j+1},
\end{equation*}
and
\begin{equation*}
    u(x_i+\Delta x, t_j) \rightarrow u_{i+1, j}.
\end{equation*}

Then we re-write the equation as

$$
    \frac{u_{i,j+1}-u_{i,j}}{\Delta t} = \frac{u_{i+1,j}+u_{i-1,j}-2u_{i,j}}{\Delta
    x^2}.
$$
Now we can define $\alpha=\Delta t / \Delta x^2$ to get

\begin{equation}
    u_{i,j+1}=\alpha u_{i-1,j} + (1-2\alpha) u_{i,j}+\alpha u_{i+1,j},
\end{equation}

Where we have discretized $x$ and $t$ so that
$$
    x_i = i\Delta x,\ i=0,1,2,...,n,
$$
and
$$
    t_j = j\Delta t,\ j=0,1,2,...,m.
$$

And this scheme is only numerically stable when the \textbf{Von Neumann
stability criterion} is met:
$$
    \alpha\leq 1/2,
$$ 
as described in (Tveito and
Winther, 2005, pp. 132-133)\cite{tveito}. And it has a \textbf{truncation
error} of
$O(\Delta x^2)$ (Tveito and Winther, 2005, p. 64).

We note that this problem can be reduced to solving a matrix system, where
$$
    A=
    \begin{bmatrix}
        1-2\alpha & \alpha & 0 & 0 & ... & 0 \\
        \alpha    & 1-2\alpha & \alpha & 0 ... & 0 \\
        0 & \alpha & 1-2\alpha & \alpha & 0 & ... \\
        ... & ... & ... & ... & ... & ... \\
        0 & ... & ... & 0 & \alpha & 1-2\alpha
    \end{bmatrix}
$$
is a tri-diagonal Topelitz matrix. And we can define a column vector for the
time step as
$$
    V_j = 
    \begin{bmatrix}
        u_{1,j}\\
        u_{2,j}\\
        ...\\
        u_{n,j}
    \end{bmatrix}
$$
so that
\begin{equation}
    V_{j+1} = AV_j.
\end{equation}


\begin{algorithm}
    \caption{Toeplitz Forward solver algorithm}\label{algo:toeplitz}
    \begin{algorithmic}
        \Require{Spacial step-size $\Delta x$}
        
        \Require{Timespan $T$ and lengthspan $L$.}
        
        \Require{Stability criterion parameter $\alpha$.}
        
        \Require{Function f(x)=u(x,0).}
        
        Calculate diagonal elements: $a=\alpha$, $b=1-2\alpha$.
    
        Calculate first timestep $u(x,0)=f(x)$ and set boundary conditions.
    
        \While{Time is less than $T$}
            
            $u(x, t+\Delta t) = a \cdot u(x-\Delta x, t) + b\cdot u(x,
            t) + a \cdot u(x+\Delta x, t)$.
    
            Set boundary conditions.
        \EndWhile
    
    \end{algorithmic}
    \end{algorithm}
    
    We will forward solve using the the algorithm in \ref{algo:toeplitz}. This is a
    method of solving a tri-diagonal matrix system without having to do the matrix
    multiplication. 
    
    For our program we have chosen to hold the entire solution in space and time in
    memory. But it is also possible to iterate in time without holding in memory
    and just save the time-steps you want to look at. But for the 1D case this is
    not a concern on a modern computer as we will look at arrays with a maximum of
    $10^4\times 10^4$, which will be under one Megabyte with 32-bit numbers.
    