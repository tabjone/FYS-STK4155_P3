% {{{
\documentclass[a4paper,10pt,english]{article}
\usepackage[utf8]{inputenc}
\usepackage[english]{babel}
\usepackage{graphicx, verbatim, amsmath, amsfonts, geometry, float, import, bm}
\usepackage{siunitx} % converts expression to SI units/notation , \num{10e-10}
\usepackage[hidelinks]{hyperref}
\usepackage{url}
\usepackage{biblatex}


\usepackage{algpseudocode}
\usepackage{algorithm}

% \bibliography{refs}

%Box around tex
\usepackage{mdframed}

%Subfigures
%\usepackage{caption, subcaption}
\usepackage{subfigure}        % imports a lot of cool and useful figure commands

\usepackage{gensymb}


% Code higligthing
\usepackage{listings}
\usepackage{xcolor}
\definecolor{codegreen}{rgb}{0,0.6,0}
\definecolor{codegray}{rgb}{0.5,0.5,0.5}
\definecolor{codepurple}{rgb}{0.58,0,0.82}

\lstdefinestyle{mystyle}{
	commentstyle=\color{codegreen},
	keywordstyle=\color{magenta},
	stringstyle=\color{codepurple},
	basicstyle=\ttfamily\footnotesize,
	breakatwhitespace=false,         
	breaklines=true,                 
	captionpos=b,                    
	keepspaces=true,                 
	numbersep=5pt,                  
	showspaces=false,                
	showstringspaces=false,
	showtabs=false,                  
	tabsize=4
}
\lstset{style=mystyle}
\setlength{\parindent}{0mm}
\setlength{\parskip}{1.5mm}

% }}}

\title{Exploring advantages/disadvantages of solving PDEs with Neural Networks}
\author{Tor-Andreas Bjone}

\begin{document}
\maketitle
\tableofcontents

\begin{abstract}                          % marks the beginning of the abstract

We looked at the mean squared error of several neural networks and an explicit
solver against a closed-form solution. We found that the neural network that
performed best was using the ReLu activation function, the RMSprop otimizer and
had 3 hidden layers. But this had a MSE of 2 orders of magnitude higher than
the explicit solver after 20 epochs. We then timed the explicit solver against a already
trained network for different matrix sizes and found that the explicit solver
was about two orders of magnitude faster than the neural network. For such a
simplistic problem a traditional solver performs better than a neural network.
\end{abstract}                            % marks the end of the abstract

\section{Introduction}
\begin{comment}
In this report we will look at ...
Motivate the reader, the first part of the introduction gives always a
motivation and tries to give the overarching ideas. What I have done. 
The structure of the report, how it is organised. Explain structure of the rapport at the end of intro. 
\end{comment}

In recent years the applications of Neural Networks have been expanded
massively. There has been done a lot of research into solving PDEs with Neural
Networks, such as the paper (Zobeiry \& Humfeld, 2020)\cite{2}.
Solving/approximating such
systems as the Schrödinger equation or the Navier-Stokes equation with no
closed-form solutions can be very beneficial. 

In this report we will tackle a simple problem, the heat equation in 1D, to see how a Neural Network
performs against a conventional numerical method, the Forward Time Central
Space (FTCS) scheme. To evaluate the methods we choose a boundary condition with a closed-form
solution and look at the difference between this and our numerical solutions.
And to set up the Neural Network we use Keras, which is a TensorFlow API.


The method section first goes trough through the closed-form solution and then
the Neural Network. And the results start by comparison of the mean squared
error (MSE) before looking at the time it takes to solve using the explicit
method versus the Neural Network.


\section{Theory and Methods}

\subsection{Heat equation}
We will look at the heat equation in one dimension.
\begin{equation*}
    \frac{\partial^2 u(x,t)}{\partial x^2} =\frac{\partial u(x,t)}{\partial t},
    t\geq 0, x\in [0,L]
\end{equation*}
or
\begin{equation*}
    u_{xx} = u_t,
\end{equation*}
with initial conditions
\begin{equation*}
    u(x,0)= \sin{(\pi x)} \hspace{0.5cm} 0 \leq x \leq L,
\end{equation*}
and with $L=1$ being the length of the $x$-region of interest. The 
boundary conditions are
\begin{equation*}
    u(0,t)= 0 \hspace{0.5cm} t \ge 0,
\end{equation*}
and
\begin{equation*}
    u(L,t)= 0 \hspace{0.5cm} t \ge 0.
\end{equation*}

\subsubsection{Closed-form solution}
The analytical solution to this problem is derived in (Tveito and
Winther, 2005, pp. 90-92)\cite{tveito}, so we will just present a small
summary of the method here. First we assume that $u(x,t)$ is linear and homogeneous and 
that the equation is separable in the form of $u_k(x,t)=X_k(x)T_k(t)$, where
$k$ refers to it being a particular solution. We can then solve the equation 
by separation of variables to find
\begin{equation*}
    u(x,t)_k=e^{-(k\pi/L)^2t}\sin{\left(\frac{k\pi}{L}x\right)},\ k=1,2,3,..
\end{equation*}
This will then give the family ${u_k}$ of particular solutions. We assume then
that $f(x)$ can be written as a linear combination of the eigenfunctions of
$X(x)$, so that
\begin{equation*}
    f(x)=\sum_{k=1}^{N} c_k \sin{\left(\frac{k\pi}{L}x\right)},
\end{equation*}
where $c_k$ is some constant. Then it follows by linearity that
\begin{equation*}
u(x,t)=\sum_{k=1}^N c_ke^{-(k\pi/L)^2t}\sin{\left(\frac{k\pi}{L}x\right)}.
\end{equation*}

Now inserting that $f(x)=\sin(\pi x)$ and $L=1$ it is easy to see that this
gives $c_1=1$ and all other constants $c_k=0$. This gives us the \textbf{closed-form
solution} of
\begin{equation*}
    u(x,t)=e^{-\pi^2 t}\sin{\left(\pi x\right)}.
\end{equation*}

\subsubsection{Numerical approximation}
We will solve this by so called FTCS-scheme (Forward Time Central Space) and
use the Forward-Euler method for moving in time. This results in

\begin{equation*}
u_t\approx \frac{u(x,t+\Delta t)-u(x,t)}{\Delta t}=\frac{u(x_i,t_j+\Delta t)-u(x_i,t_j)}{\Delta t}
\end{equation*}
and
\begin{equation*}
u_{xx}\approx \frac{u(x+\Delta x,t)-2u(x,t)+u(x-\Delta x,t)}{\Delta x^2},
\end{equation*}
or
\begin{equation*}
    u_{xx}\approx \frac{u(x_i+\Delta x,t_j)-2u(x_i,t_j)+u(x_i-\Delta x,t_j)}{\Delta x^2}.
\end{equation*}
And to simplify notation, let
\begin{equation*}
    u(x_i, t_j + \Delta t)\rightarrow u_{i,j+1},
\end{equation*}
and
\begin{equation*}
    u(x_i+\Delta x, t_j) \rightarrow u_{i+1, j}.
\end{equation*}

Then we re-write the equation as

$$
    \frac{u_{i,j+1}-u_{i,j}}{\Delta t} = \frac{u_{i+1,j}+u_{i-1,j}-2u_{i,j}}{\Delta
    x^2}.
$$
Now we can define $\alpha=\Delta t / \Delta x^2$ to get

\begin{equation}
    u_{i,j+1}=\alpha u_{i-1,j} + (1-2\alpha) u_{i,j}+\alpha u_{i+1,j},
\end{equation}

Where we have discretized $x$ and $t$ so that
$$
    x_i = i\Delta x,\ i=0,1,2,...,n,
$$
and
$$
    t_j = j\Delta t,\ j=0,1,2,...,m.
$$

And this scheme is only numerically stable when the \textbf{Von Neumann
stability criterion} is met:
$$
    \alpha\leq 1/2,
$$ 
as described in (Tveito and
Winther, 2005, pp. 132-133)\cite{tveito}. And it has a \textbf{truncation
error} of
$O(\Delta x^2)$ (Tveito and Winther, 2005, p. 64).

We note that this problem can be reduced to solving a matrix system, where
$$
    A=
    \begin{bmatrix}
        1-2\alpha & \alpha & 0 & 0 & ... & 0 \\
        \alpha    & 1-2\alpha & \alpha & 0 ... & 0 \\
        0 & \alpha & 1-2\alpha & \alpha & 0 & ... \\
        ... & ... & ... & ... & ... & ... \\
        0 & ... & ... & 0 & \alpha & 1-2\alpha
    \end{bmatrix}
$$
is a tri-diagonal Topelitz matrix. And we can define a column vector for the
time step as
$$
    V_j = 
    \begin{bmatrix}
        u_{1,j}\\
        u_{2,j}\\
        ...\\
        u_{n,j}
    \end{bmatrix}
$$
so that
\begin{equation}
    V_{j+1} = AV_j.
\end{equation}


\begin{algorithm}
    \caption{Toeplitz Forward solver algorithm}\label{algo:toeplitz}
    \begin{algorithmic}
        \Require{Spacial step-size $\Delta x$}
        
        \Require{Timespan $T$ and lengthspan $L$.}
        
        \Require{Stability criterion parameter $\alpha$.}
        
        \Require{Function f(x)=u(x,0).}
        
        Calculate diagonal elements: $a=\alpha$, $b=1-2\alpha$.
    
        Calculate first timestep $u(x,0)=f(x)$ and set boundary conditions.
    
        \While{Time is less than $T$}
            
            $u(x, t+\Delta t) = a \cdot u(x-\Delta x, t) + b\cdot u(x,
            t) + a \cdot u(x+\Delta x, t)$.
    
            Set boundary conditions.
        \EndWhile
    
    \end{algorithmic}
    \end{algorithm}
    
    We will forward solve using the the algorithm in \ref{algo:toeplitz}. This is a
    method of solving a tri-diagonal matrix system without having to do the matrix
    multiplication. 
    
    For our program we have chosen to hold the entire solution in space and time in
    memory. But it is also possible to iterate in time without holding in memory
    and just save the time-steps you want to look at. But for the 1D case this is
    not a concern on a modern computer as we will look at arrays with a maximum of
    $10^4\times 10^4$, which will be under one Megabyte with 32-bit numbers.
    
\section{Results and Discussion}

\begin{figure}[h!]
    \centering
    \includegraphics[width=8cm]{../Figures/analytic_solution.png}
    \caption{Analytic solution for the heat equation with initial conditions
    $u(x,0)=\sin{(\pi x)}$ and boundary conditions $u(0,t)=u(1,t)=0$.}
    \label{fig:analytic_solution}
\end{figure}

\textit{All networks are initialized with weights drawn from a Xavier uniform
initializer and bias=0}.
\\~\\
In figure \ref{fig:analytic_solution} we see the analytic solution for our
given boundary and initial conditions. This will be used to test our numerical
solvers using the absolute error between this and the numerical approximation.
This can be seen in figure \ref{fig:abs_error_all}. In the subfigures (a)-(d)
we see the NNs prediction for different activation functions and optimizers. We
can see that ReLu with RMSprop has the lowest error. And we note that all of
the approximations have the highest loss at the boundaries. We can especially
see that Adam with ReLu has a very high error at the $t=0$ boundary. The reason
our network is not handling the boundaries well is because the loss function
has no special penalizing of wrong boundaries. This could be fixed by creating
a new loss function by looking at the minimization of the NNs prediction

\begin{equation*}
    \min_{\hat{u}} \left\{
    \lVert \hat u_{xx} - \hat u_t \rVert^2 
    + \lVert \hat u(0,t) \rVert^2 
    + \lVert \hat u(1,t) \rVert^2
    + \lVert \hat u(x,0)-\sin{(\pi x)}\rVert^2 \right\},
\end{equation*}
where this norm $\lVert \cdot \rVert$ is to be read as the grand sum of the
matrix. We then define our cost/loss-function as
\begin{equation*}
    C(\hat u) =\frac{1}{2} \left[ \lVert \hat u_{xx} - \hat u_t \rVert^2 
    + \lVert \hat u(0,t) \rVert^2 
    + \lVert \hat u(1,t) \rVert^2
    + \lVert \hat u(x,0)-\sin{(\pi x)}\rVert^2 \right].
\end{equation*}
We can then add parameters in front of the norms to decide how costly it would
be for the network to not prioritize the boundaries. This is called a physics
informed neural network and has been explored in a previous paper (Zobeiry \&
Humfeld, 2020)\cite{2}. Getting back to figure \ref{fig:abs_error_all}, we look
at the explicit solver in (e). Note here that the limits on the colorbar is
changed because the error is so small compared to that of the NNs. We see that
now our boundaries is forced to be correct. And we can see that as we move in
time the middle of the rod gets a larger and larger error until it starts
cooling down and of course the error gets smaller because the values get closer
and closer to zero because of the $u(0,t)=u(1,t)=0$ boundaries. We look more at
the explicit solver in \ref{fig:MSE_b}. We can see a big drop in MSE as half
$\alpha$ from the criterion $1/2$ to $1/4$. By doing this we decrease the MSE
by about 4 orders of magnitude. While decreasing $\Delta x$ by $10$ we only get
a decrease in MSE of about one order of magnitude. This means that decreasing
$\alpha$ will have the biggest effect on the accuracy of the solution. Now
looking at \ref{fig:loss_vs_epochs} we can see loss versus epochs for different
activation functions and optimizers. We see that the MSE is higher for all of
these compared to the explicit solver. And we see that Adam with ReLu looks
constant. This could be because the network died or it reaches a local minima
in the gradient decent.



\begin{figure}[h!]
    \centering
    \includegraphics[width=8cm]{../Figures/b_mse.png}
    \caption{MSE of explicit solver for different $\alpha=\Delta t/\Delta x^2$
    and $\Delta x$ values.}
    \label{fig:MSE_b}
\end{figure}

\begin{figure}[h!]
    \centering
    \includegraphics[width=8cm]{../Figures/loss_vs_epochs.png}
    \caption{MSE loss function for different optimizers and activation functions.
    All are run with a batch size of 30 and trained on a $(n=100)\times
    (m=100)$ matrix against the analytical solution. The networks have 3 hidden
    layers, one of 16 neurons and two of 32 neurons.}
    \label{fig:loss_vs_epochs}
\end{figure}

We further explore RMSprop with ReLu since this has the lowest MSE. In figure
\ref{fig:loss_vs_layers} we see the MSE as a function of epochs for different
number of hidden layers. We see that for 5 hidden layers we have a constant
loss. This could be because the network is too complex for our problem and
dies. This may be solved with different initializations of weights and biases.
But it seems like for 3 and 4 hidden layers we do not get a big difference in
MSE so 3 hidden layers may be enough for such a simple problem. 

\begin{figure}[h!]
    \centering
    \includegraphics[width=8cm]{../Figures/loss_vs_layers.png}
    \caption{MSE loss function for different number of hidden layers. First
    hidden layer is 16 neurons and the rest are 32. 
    All are run with a batch size of 30 and trained on a $(n=100)\times
    (m=100)$ matrix against the analytical solution. Activation functions and
    ReLu and the optimizer is RMSprop.}
    \label{fig:loss_vs_layers}
\end{figure}

We chose our 3 layer network with RMSprop activation function and Adam
optimizer and train this. Then in figure \ref{fig:time_vs_size} we see the time it takes to
predict depending on the matrix size against the time used by the explicit
solver. We can see that in logspace both graphs are linear, meaning they are
taking exponentially more time as we increase the matrix size. This is
expected for the explicit solver. For the Neural Network we disregard the low
matrix size points because this is only run one time and is therefore very
dependent on small variations by the processes already running on the computer.
These effects will be smaller and smaller the bigger the matrix size as such
effects will average out. So then we can also say that the predictions of the
neural networks grow linearly in logspace. And it is about two orders of
magnitude bigger than the explicit solver.

\begin{figure}[h!]
    \centering
    \includegraphics[width=8cm]{../Figures/time_vs_matrix_size.png}
    \caption{Time vs matrix size for a neural network with three hidden layers,
    one of 16 neurons and two of 32 neurons. The network is already trained
    using a batch size of 30 on a $(n=100)\times
    (m=100)$ matrix against the analytical solution. Activation functions and
    ReLu and the optimizer is RMSprop. And the explicit solver has $\alpha=1/4$
    and uses different $\Delta x$ to calculate the matrix size that the NN gets
    to solve. Run on a 1,8 GHz dual-core Intel Core i5.}
    \label{fig:time_vs_size}
\end{figure}

\begin{comment}
\begin{figure}
\centering
\subfigure[Rms, ReLu]{\label{fig:a}\includegraphics[width=70mm]{../Figures/error_rms_relu.png}}
\subfigure[Rms, Sigmoid]{\label{fig:b}\includegraphics[width=70mm]{../Figures/error_rms_sigmoid.png}}
\subfigure[Adam, ReLu]{\label{fig:c}\includegraphics[width=70mm]{../Figures/error_adam_relu.png}}
\subfigure[Adam, Sigmoid]{\label{fig:d}\includegraphics[width=70mm]{../Figures/error_adam_sigmoid.png}}
\subfigure[Explicit solver]{\label{fig:e}\includegraphics[width=70mm]{../Figures/error_explicit.png}}
\caption{Absolute error for NN in (a)-(d) and explicit solver in (e). NN is
trained with 20 epochs and a batch size of 30 with $(n=100)\times (m=100)$
data points of $x$
and $t$ respectively, using the analytical solution for MSE as the loss
function. The networks have three hidden layers, one of 16 neurons and two of 32.
The explicit solver in (e) is using $\Delta x=0.1$ and $\alpha=1/4$.}
\label{fig:abs_error_all}
\end{figure}
\end{comment}



Possible future work: create cost function based on minimization of the heat eq



\section*{References}  
\begin{itemize}
\bibitem[1]{1} Tveito, A. \& Winther, R. (2005) Introduction to Partial Differential Equations: A computational approach. 2nd edn. New York, NY, USA: Springer.

\bibitem[2]{2} Humfeld, K \& Zobeiry, N. (2020)  Physics-Informed Machine Learning Approach for Solving Heat Transfer Equation
    in Advanced Manufacturing and Engineering Applications. Washington, Seattle, WA: University of Washington
\end{itemize}



\begin{figure}
\centering
\subfigure[Rms, ReLu]{\label{fig:a}\includegraphics[width=70mm]{../Figures/error_rms_relu.png}}
\subfigure[Rms, Sigmoid]{\label{fig:b}\includegraphics[width=70mm]{../Figures/error_rms_sigmoid.png}}
\subfigure[Adam, ReLu]{\label{fig:c}\includegraphics[width=70mm]{../Figures/error_adam_relu.png}}
\subfigure[Adam, Sigmoid]{\label{fig:d}\includegraphics[width=70mm]{../Figures/error_adam_sigmoid.png}}
\subfigure[Explicit solver]{\label{fig:e}\includegraphics[width=70mm]{../Figures/error_explicit.png}}
\caption{Absolute error for NN in (a)-(d) and explicit solver in (e). NN is
trained with 20 epochs and a batch size of 30 with $(n=100)\times (m=100)$
data points of $x$
and $t$ respectively, using the analytical solution for MSE as the loss
function. The networks have three hidden layers, one of 16 neurons and two of 32.
The explicit solver in (e) is using $\Delta x=0.1$ and $\alpha=1/4$.}
\label{fig:abs_error_all}
\end{figure}



\end{document}
